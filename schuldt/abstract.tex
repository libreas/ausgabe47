\textbf{Zusammenfassung}: Die Waldhandschrift wurde 1985--1987 nach dem
Vorbild mittelalterlicher Handschriften (auf Pergament, mit
mittelalterlichen Techniken des Schreibens und Buchmalens) produziert.
Thematisch geht es in ihr um das Waldsterben. Sie unterliegt
spezifischen Nutzungsbedingungen, unter anderem darf sie nur per Hand
abgeschrieben werden. Heute liegt sie in der Stiftsbibliothek
St.~Gallen. In diesem Essay geht es darum, die Waldhandschrift möglichst
nahe am Objekt zu beschreiben und gleichzeitig eine Interpretation zu
versuchen: Wie wirkt sie heute?

\begin{center}\rule{0.5\linewidth}{0.5pt}\end{center}

\textbf{Abstract}: The Waldhandschrift (forest manuscript) is a book
produced 1985--1987 following medieval procedures (on parchment, with
medieval techniques for writing and book illustration). Its topic is the
“Waldsterben” (forest decline). It is subject to specific
requirements, for instance handwriting as the only form of reproduction
method permitted. Today, the Waldhandschrift is deposited at the abbey
library of Saint Gall. This essay tries to describe the Waldhandschrift
as concrete as possible and attempts to interpret this manuscript: How
does it affect today?
